\documentclass[12pt]{article}
\usepackage{dirtree}
\usepackage{graphicx}
\usepackage{listings}
\usepackage{hyperref}

\title{Ex Mente Python Assignment}
\author{Nicholas Holtzhausen}
\date{February 2024}

\begin{document}

\maketitle

\section{Introduction}
The purpose of this project is to assess the programming capabilities of a
junior developer. The project entails creating a Python package with file
operation and data analysis functionalities. The developer must also follow
proper version control and documentation practices. This includes creating a
package that can be installed and used easily, as well as writing meaningful
functions that are well-documented and function properly. Proper branching
techniques must also be used, and the developer must write clear and concise
commit messages.

\section{Implementation}
The \verb|DataAnalysis| package is designed to be deployed as a pip package, so
that it can be downloaded and installed easily and publically. The package
functions are contained in the \verb|app| module, which is in the
\verb|DataAnalysis| directory, along with an \verb|__init__.py| file, which
handles the imports for the package. In the base directory there is also a
\verb|setup.py| file that handles all the functionality for the package being
uploaded to PyPi.\newline\newline
As mentioned before, the functions are all contained in the \verb|app| module,
and can be called as follows: \verb|app.name_of_function(arg1, arg2)|, after
importing the functions: \verb|from DataAnalysis import *|.\newline\newline
Thanks to the simplicity of \verb|pandas|, file input and output can be
implemented with ease, using functions such as \verb|pd.read_csv| and
\verb|pd.to_csv|. The \verb|subprocess| library is also easy to use, and allows
for the simple creation of child processes on the command line with the
\verb|run()| function.

\section{Challenges}
In the initial stages of the project, it was difficult to determine how the
package should be structured. Once I realised that the existence of a class is
unnecessary, the development process became simpler.

\section{Version Control}
For the version control, the development workflow was used. This method
stipulates that changes to the codebase should only be made on the development
branch, or on branches that originated from the development branch, such as a
\verb|bug_fix| branch. The development branch is only merged into the main
branch when there is a fully functional and complete version on the development
branch that is ready for deployment.\newline\newline
Other than the development branch, there is also a \verb|documentation| branch,
which originates from the main branch and is exclusively for amendments to the
source code of the report, the README, and any other
documentation.\newline\newline
This method allowed for a modularised approach when editing any source code, and
caused almost no merge conflicts. In addition to the use of branches, tags were
also used to keep track if the official versions of the software.

\section{Data Analysis}

\section{Conclusion}

\end{document}